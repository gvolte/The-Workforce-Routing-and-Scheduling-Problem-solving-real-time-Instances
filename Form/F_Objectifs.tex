
Comme le nombre de tâches est bien plus grand que le nombre de techniciens, il y aura donc des tâches non affectées.
\begin{enumerate}
\item \label{item:Funcobj1} Un des objectifs sera donc de maximiser le nombre de tâches affectées en fonction de leur priorité, on peut aussi minimiser le complémentaire (i.e. minimiser le nombre de tâches non affectées).
\item  \label{item:Funcobj2} Il faut aussi minimiser le retard des tâches par rapport à la date limite/ le nombre de tâches effectuées en retard.



\item  \label{item:Funcobj3} Par souci d'efficacité on souhaite minimiser la durée des transports pour que le temps de travail des techniciens soit utilisé pour effectuer des tâches.
\item \label{item:Funcobj4}On souhaite enfin maximiser ou minimiser la différence de compétence entre techniciens et tâches en fonction du problème : 
\begin{itemize}
\item si l'on maximise cette différence, on va essayer d'affecter des techniciens très qualifiés à des tâches peu contraintes. Cela permet d'augmenter les chances de résolution de la tâche du premier coup.
\item si on la minimise on va essayer d'affecter les techniciens qui ont juste les compétences requises par la tâche, et ainsi optimiser les coûts.
\end{itemize}
\end{enumerate}


\indent On note $\task^* \subseteq \task$ l'ensemble des tâches planifiées, $\task_t^* \subseteq \task^*$ l'ensemble des tâches planifiées pour le technicien $t$, $m_{i,j}$ la distance entre la position $i$ et la position $j$ et $D_j = max(0,C_j-d_j)$\footnote{Pour plus d'information sur les notations utilisées, se référer au livre \cite{blazewicz2007handbook}.}.
\\


La fonction objectif multi-critère est définie comme suit :

\begin{itemize}
\item Max $\sum\limits_{j \in \task^*} \omega_{j}$ ou Min $\sum\limits_{j \in \task \backslash \task^*} \omega_{j}$(\hyperref[item:Funcobj1]{objectif~\ref*{item:Funcobj1}}) .
\\
\item Min $\sum\limits_{j \in \task^*} D_j$ (\hyperref[item:Funcobj2]{objectif~\ref*{item:Funcobj2}}) .
\\
\item Min $\sum\limits_{i \in \tech}\sum\limits_{j \in \task_i^*} \dist{i}{j}$ (\hyperref[item:Funcobj3]{objectif~\ref*{item:Funcobj3}}) .
\\
\item Max/Min $\sum\limits_{{\tech_i} \in \tech}\sum\limits_{{\task_j} \in \task_t^*}\sum\limits_{k=1}^\nskill  \gapS{\tech_i}{\task_j}{k} $ (\hyperref[item:Funcobj4]{objectif~\ref*{item:Funcobj4}}) .
\end{itemize}


